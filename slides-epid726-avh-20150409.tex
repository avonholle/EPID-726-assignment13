% slides-epid726-avh-20150406.tex

%\documentclass[10pt, handout	]{beamer}\usepackage[]{graphicx}\usepackage[]{color}
\documentclass[10pt]{beamer}\usepackage[]{graphicx}\usepackage[]{color}

%
%% BEGINNING of code from http://tex.stackexchange.com/questions/47747/condense-beamer-notes-into-one-page 
%% ------------------------------------------------------------
%\newif\ifshowonlynotes
%\showonlynotestrue
%%\showonlynotesfalse
%
%\makeatletter
%\newif\ifbeamer@inlecture\beamer@inlecturetrue
%\def\beamer@currentmode{beamer}
%\input{beamerbasenotes.sty}
%\def\beamer@currentmode{article}
%
%\renewcommand\beamer@outsideframenote[2][]{%
  %\def\beamer@noteenvstart{}%
  %\def\beamer@noteenvend{}%
  %\setkeys{beamernotes}{#1}%
  %\par
  %\beamer@noteenvstart#2\beamer@noteenvend%
  %\par
%}
%
%\setbeamertemplate{frame begin}{\beamer@framenotesbegin}
%\setbeamertemplate{frame end}{\beamer@setupnote\beamer@notesactions}
%
%\ifshowonlynotes
  %\let\beamer@dosingleframe=\beamer@donoframe
  %\g@addto@macro\beamer@endframe{\usebeamertemplate{frame end}}
%\fi
%\makeatother
%% ------------ END of code from http://tex.stackexchange.com/questions/47747/condense-beamer-notes-into-one-page

\note{USE THIS TO PRINT OFF NOTES..............}
%\setbeameroption{show notes} %un-comment to see the notes
\setbeameroption{show only notes}

\definecolor{shadecolor}{rgb}{.97, .97, .97}
\definecolor{messagecolor}{rgb}{0, 0, 0}
\definecolor{warningcolor}{rgb}{1, 0, 1}
\definecolor{errorcolor}{rgb}{1, 0, 0}
\newenvironment{knitrout}{}{} % an empty environment to be redefined in TeX

%\usetheme{Madrid}
%\usetheme{PaloAlto}
%\usetheme{Dresden}
%\usetheme{Singapore}
\usetheme{Frankfurt}
%\usetheme{Szeged}
\usecolortheme{beetle}
%\usecolortheme{tarheel2}
%\usefonttheme[stillsansserifsmall]{serif}

%\usepackage[english]{babel}
%\usepackage[latin1]{inputenc}
%\usepackage[orientation=landscape, size=custom,width=142.24, height=91.44, scale=1.7]{beamerposter} % these match 56 by 36 inches -- ARIC template
%\usepackage[orientation=landscape, size=a0, scale=1.5]{beamerposter}
%\usepackage[numbers,sort]{natbib}
\usepackage{alltt}
\usepackage{amsmath}

\usepackage{array,booktabs,tabularx}
%\usepackage[skip=0pt]{caption} %caption does not recognize beamer. don't use
\usepackage{graphicx,wrapfig,lipsum}
\usepackage{changepage}
\usepackage{booktabs}
\usepackage[font=small,begintext=\textquotedblleft,endtext=\textquotedblright]{quoting} % this option sets font size of quotes automatically.
\usepackage{pifont}
%\usepackage{caption}
%\usepackage{pgf,tikz} % see http://codealamode.blogspot.com/2013/06/drawing-dags-latex-solution.html
% for some reason if I exclude tikz this works. is there a conflict with some other package?
%\usetikzlibrary{matrix, shapes, arrows, positioning, chains}
%\usepackage[T1]{fontenc}
%\usepackage{arev}

\setbeamersize{text margin left=10pt,text margin right=10pt} %set margin sizes

\note{Print this off before lecture}
%\usepackage{handoutWithNotes} % see http://tex.stackexchange.com/questions/4765/printing-beamer-slides
%% \pgfpagesuselayout{4 on 1}[a4paper,landscape,border shrink=5mm
%\pgfpagesuselayout{4 on 1 with notes}[a4paper,border shrink=5mm]

% see http://tex.stackexchange.com/questions/12703/how-to-create-fixed-width-table-columns-with-text-raggedright-centered-raggedlef
\usepackage{array}
\newcolumntype{L}[1]{>{\raggedright\let\newline\\\arraybackslash\hspace{0pt}}m{#1}}
\newcolumntype{C}[1]{>{\centering\let\newline\\\arraybackslash\hspace{0pt}}m{#1}}
\newcolumntype{R}[1]{>{\raggedleft\let\newline\\\arraybackslash\hspace{0pt}}m{#1}}

% got from http://tex.stackexchange.com/questions/48023/mimic-bibtex-apalike-with-biblatex-biblatex-apa-broken
\PassOptionsToPackage{
        style=numeric,
        hyperref=true,
        backend=bibtex,
        maxbibnames=99,
        firstinits=true,
        uniquename=init,
        maxcitenames=2,
        parentracker=true,
        url=false,
        doi=true,
        isbn=false,
        eprint=false,
        backref=true,
            }   {biblatex}

\usepackage{biblatex}
\addbibresource{bib1}
% see http://tex.stackexchange.com/questions/43083/author-year-abbr-journal-name-as-citation-style
\DeclareCiteCommand{\longcite}{(}{%
    \printnames[author]{author}, \printfield{year})}{}{}%

\newcommand{\customcite}[1]{\citeauthor{#1}, \citetitle{#1}, \citeyear{#1}}
\newcommand{\customcitetwo}[1]{\citeauthor{#1}, \citeyear{#1}}

\renewcommand{\footnotesize}{\tiny}  % change font size of citation
\renewcommand\multicitedelim{\addsemicolon\space} % This doesn't seem to work, but see http://tex.stackexchange.com/questions/167665/multiple-references-with-footfullcite


% This is based on the template at http://www-i6.informatik.rwth-aachen.de/~dreuw/latexbeamerposter.php

%%%%%%%%%%%%%%%%%%%%% Edit this section with your info %%%%%%%%%%%%%%%%%%%%%%%
% see http://tex.stackexchange.com/questions/9740/how-can-i-add-vertical-space-to-a-beamercolorbox-to-make-it-align-with-another-o
\title[Infant growth trajectories and adolescent HDL-C]{Infant growth trajectories and high density lipoprotein cholesterol levels in adolescence. \rule[-1\normalbaselineskip]{0pt}{0pt}}
%\title{\setlength\lineskip{20pt}Childhood BMI associated with low HDL-C levels in adolescence in a Chilean cohort}
%\title{Childhood BMI associated with low HDL-C levels in adolescence in a Chilean cohort}
\author[vonholle@email.unc.edu]{\small Ann Von Holle}
%\author{\small Ann Von Holle}
%\titlegraphic{\includegraphics[scale=0.25]{logoblackontranspsmall.png}\includegraphics[scale=0.1]{gl-2-hsci}} %this is the path to your logo

%%%%%%%%%%%%%%%%%%%%%%%%%%%%%%%%%%%%%%%%%%%%%%%%%%%%%%%%%%%%%%%%%%%%%%%%%%%%
\IfFileExists{upquote.sty}{\usepackage{upquote}}{}
\newcommand\Fontvi{\fontsize{6}{7.2}\selectfont}

\setbeamertemplate{itemize items}[square]

\AtBeginSection[] % http://tex.stackexchange.com/questions/28654/beamer-table-of-contents-display-all-subsections-below-section
{
\begin{frame}<beamer>{Table of Contents}
\tableofcontents[currentsection,%currentsubsection, 
    hideallsubsections, 
		sectionstyle=show/shaded,
]
\end{frame}
}

\begin{document}

\setbeamertemplate{caption}{\insertcaption}

\begin{frame}
\titlepage
\centering
\end{frame}

\section{Significance and Innovation}

\note{I will start with some background on the outcome of interest and theory informing my aims.}

\subsection{CVD}
\begin{frame}
\frametitle{Cardiovascular disease (CVD) and high density lipoprotein cholesterol (HDL-C)}
\begin{itemize}
\item Cardiovascular disease (CVD) is a chronic disease responsible for one out of every three deaths in the United States.
\smallskip
\item Primary and modifiable risk factors including blood pressure, smoking, physical inactivity, obesity, and blood cholesterol. 
\smallskip
\item High density lipoprotein cholesterol (HDL-C) is one of the blood cholesterols.
	\begin{itemize}
		\item A strong risk factor for CVD.
		\item Function in the development of CVD remains unknown.
		\item Investigation remains active to determine both HDL function in atherosclerosis and potential areas for intervention such as pharmacotherapy.
		\end{itemize}
	\end{itemize}
\end{frame}


\subsection{DOoHD}
\begin{frame}
\frametitle{Developmental Origins of Health and Disease}
\centering
\includegraphics[scale=0.25]{hanson-fig1}
\footfullcite{hanson_early_2014}
\note{This figure is a lifecourse view/perspective of noncommunicable disease risk.}
\note{DOHaD proposes that a wide range of environmental conditions during embryonic development and early life determine susceptibility to disease during adult life.}
\note{Early life events influence chronic disease susceptibility and as a result, best point to intervene is in early life.}
\end{frame}

\subsection{Postnatal growth}
\begin{frame}
\frametitle{Postnatal growth and HDL-C}

\begin{itemize}
	\item Repeated animal and human studies have shown that postnatal growth is associated with the later development of adverse cardiovascular risk factors like low HDL-C levels.
\end{itemize}

\begin{table}[H]
\begin{tabular}{llC{2cm}C{2cm}C{2cm}C{2cm}}
\toprule
 & Author     & Year published & Direction of growth with increase in HDL-C & 2+ observations in change measure & Non-European sample? \\
\midrule
1 & Corvalan   & 2009           & +                                    &              & \ding{52}               \\
2 & Ekelund    & 2007           & +      	                             &               &               \\
3 & Howe       & 2010           & --                              &               &               \\
4 & Kajantie   & 2008           & +                                    &               &               \\
5 & Leunissen  & 2009           & --                              &               &               \\
6 & Oostvogels & 2014           & --                              &               &               \\
7 & Tzoulaki   & 2010           & +                                    & \ding{52}    &              \\
\bottomrule
\end{tabular}
\end{table}
\smallskip
%	\item Most studies addressing the hypothesis of growth effects on HDL-C are limited in their characterization of growth and predominantly European and homogenous in racial and ethnic composition. 
%\end{itemize}
\end{frame}
\note{These studies have persisted across different cohort years.}


\begin{frame}
\frametitle{Postnatal growth as an environmental cue}
\begin{itemize}
	\item Abnormal postnatal growth hypothesized to alter liver growth and subsequent HDL metabolism\footfullcite{kajantie_growth_2008, perala_early_2012}.
\end{itemize}
\centering
\includegraphics[scale=0.11]{fig-timeline-rev}
\end{frame}

\section{Specific Aims}

\subsection{Aim 1}
\begin{frame}
\frametitle{Aims}
Overall: 
\begin{itemize}
	\item Investigate the association between postnatal growth trajectories and HDL-C levels in adolescence
		\begin{itemize}
			\item Contemporary Chilean birth cohort with monthly measures of weight in the first year of life 
			\item High quality clinical measures of cardiovascular disease risk factors.
			\end{itemize}
		\end{itemize}
\smallskip

\textit{Note: Unless otherwise indicated, postnatal growth trajectories are in terms of a weight-for-length outcome.}

\end{frame}

\begin{frame}
\frametitle{Aim 1}
What do growth trajectories look like for infants from 0 to 12 months and what are some significant predictors?
\smallskip
\begin{description}
	\item [Aim 1]  Characterize individual growth trajectories in the first year of life and replicate predictors of growth using external validation with an independent sample. 
\end{description}	

\begin{itemize} 
\item We expect to replicate previous findings\footfullcite{pizzi_prenatal_2014} indicating a positive association between:
	\begin{enumerate} 
		\item Maternal characteristics such as pre-pregnancy BMI, height and age with trajectory size.
		\item Maternal education and trajectory velocity.
		\end{enumerate}
\end{itemize}
\end{frame}

\subsection{Aim 2}
\begin{frame}
\frametitle{Aim 2}
Are there any specific types of postnatal growth trajectories associated with adverse levels of HDL-C?
\smallskip

\begin{description}
	\item[Aim 2] Examine the association between postnatal growth trajectories and HDL-C levels.
\end{description}

\begin{itemize} 
\item Based on prior evidence we expect
	\begin{enumerate} 
		\item Infants with steeper growth trajectories to associate with adverse HDL-C levels in adolescence.
		\item Males will show a stronger association between size, tempo and velocity measures than females.
		\end{enumerate}
\end{itemize}
\end{frame}

\subsection{Aim 3}
\begin{frame}
\frametitle{Aim 3}
Do growth trajectories modify the association between genetic variants related to HDL metabolism and HDL-C levels in adolescence?
\begin{description}
	\item[Aim 3] Assess gene-environment interaction between growth trajectory characteristics and genetic variants of HDL metabolism with HDL-C at 17 years of age as an outcome.
\end{description}

\begin{itemize}
\item We expect that
	\begin{enumerate} 
		\item The selected genetic variants will associate with HDL-C levels in adolescence.
		\item A gene-environment interaction exists in which extreme and less favorable growth characteristics will exhibit stronger, deleterious associations between the genetic variants and HDL-C levels.
		\end{enumerate}
\end{itemize}

\end{frame}


\section{Approach}

\subsection{SITAR}
\begin{frame}
\frametitle{SITAR method}
SITAR: SuperImposition by Translation And Rotation\footfullcite{beath_infant_2007,cole_sitar--useful_2010}.\\
\smallskip
Use SITAR to measure three measures of postnatal growth: size, tempo and velocity.
\smallskip

\centering
\includegraphics[scale=0.35]{sim-growth-2}
\end{frame}

\subsection{LGMM}
\begin{frame}
\frametitle{Latent growth mixture models (LGMM)}
Use LGMM to identify underlying patterns in infant growth and regress HDL-C levels on these latent groups.

\centering
\includegraphics[scale=0.4]{sim-lgmm-2}
\end{frame}

\section{Strength and Limitations}

\subsection{Strength and Limitations}
\begin{frame}

\underline{Strengths}:
\begin{description}
	\item[Underepresented sample] The majority of studies on this topic to date involve European populations.
	\item[Detail] This project uses methods unavailable to most prior studies due to limited number of postnatal growth observations.
%	\item[Gene-environment interaction] No study has examined gene-environment interactions relating to HDL metabolism and postnatal growth.
	\end{description}
	
\underline{Limitations}:
\begin{description}
	\item[Outcome a proxy] Many factors contribute to postnatal growth trajectories. Infants with similar trajectories could experience different prior exposures and different risk later in life\footfullcite{hanson_early_2014}.
	\item[Washout] Cumulative exposures to adverse life circumstances over the life course wash out any effect present in infancy\footfullcite{valente_relation_2015}.
%	\item[Homogeneity of growth] Most mothers in sample breastfed in the first six months and growth may be more homogenous than samples with more formula feeding.
	\end{description}
\end{frame}

\section{Public Health Implications}
\subsection{Public Health Implications}
\begin{frame}
Results from this research can:
\begin{itemize}
\item Inform efforts to identify predictors of HDL-C and its accompanying risk of CVD
\item Support of postnatal growth as an environmental cue inducing permanent change of HDL metabolism. 
\end{itemize}

\mediumskip
Further attention would be warranted to investigate the composition of optimal postnatal growth. These interventions would have the potential for modification of CVD later in life.

\end{frame}

\begin{frame}
Special thanks to my
\begin{itemize}
	\item In-class reviewers: Nelson Pace and Sydney Jones
	\item Group leader: Dr. Julie Daniels
	\end{itemize}
	
	Web page: vonholle.web.unc.edu
\end{frame}
\end{document}